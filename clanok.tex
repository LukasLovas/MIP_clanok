\documentclass{article}

\usepackage[english]{babel}


\usepackage[letterpaper,top=2cm,bottom=2cm,left=3cm,right=3cm,marginparwidth=1.75cm]{geometry}


\usepackage{amsmath}
\usepackage{graphicx}
\usepackage[colorlinks=true, allcolors=blue]{hyperref}

\title{Porovnávanie hier podporujúcich výučbu a učenie}
\author{Lukáš Lovás}
\begin{document}
\maketitle

\section{Spresnenie rámcovej témy}
V dnešnej dobe je pre mládež populárnou témou hranie hier. Z tohoto dôvodu vznikol smer výskumu, ktorý sa zameriava na využitie hier pri výučbe a učení v školách, pričom sa kladie dôraz na efektivitu výučby a učenia. V mojom článku sa chcem venovať niekoľko programom určených práve pre školy a študentov na zefektívnenie učenia, pričom mojim hlavným cieľom je porovnať konkrétne hry a ich vplyv na efektivitu učenia a výučby programovania. Výstupom bude prehľadový článok porovnávajúci konktrétne hry, cez vybrané aspekty. Medzi vybrané hry, ktoré som sa rozhodol analyzovať, patrí Clara's World\cite{ClarasWorld}, jednoducho vyzerajúca hra ktorá má hráčov oboznámiť so základnou syntaxou programovania, znázorňovaného pomocou Clary - herného modelu lienky, ktorý plní zadané príkazy pohybovaním sa po hernej ploche. ClaraQuest\cite{ClaraQuest} je modernejšia verzia, ktorá podporuje VR platformu, pričom bola vyvinutá v rámci výskumu na FIIT STU. Ďaľšou skúmanou hrou je Cubely, ktorá sa taktiež zameriava na podporu výučby objektovo orientovaného programovania. A poslednou skúmanou hrou je Minecraft Education, ktorá vznikla z populárnej hry Minecraft. Podporované hry využívajú aj aspekty virtuálnej reality, čím sa stávajú lepšie porovnatelnými medzi sebou. V závere článku zhodnotím výhody a nevýhody jednotlivých hier a sofrmulujem špecifické odporúčania pre zefektívnenie učenia a výučby. 

\bibliographystyle{alpha}
\bibliography{sample}

\end{document}