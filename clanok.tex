% Metódy inžinierskej práce

\documentclass[10pt,oneside,slovak,a4paper]{article}

\usepackage[slovak]{babel}
%\usepackage[T1]{fontenc}
\usepackage[IL2]{fontenc} % lepšia sadzba písmena Ľ než v T1
\usepackage[utf8]{inputenc}
\usepackage{graphicx}
\usepackage{url} % príkaz \url na formátovanie URL
\usepackage{hyperref} % odkazy v texte budú aktívne (pri niektorých triedach dokumentov spôsobuje posun textu)

\usepackage{cite}
%\usepackage{times}

\pagestyle{headings}

\title{Porovnávanie hier podporujúcich výučbu a učenie}

\author{Lukáš Lovás\\[2pt]
	{\small Slovenská technická univerzita v Bratislave}\\
	{\small Fakulta informatiky a informačných technológií}\\
	{\small Semestrálny projekt v predmete Metódy inžinierskej práce}\\
	{\small  ak. rok 2022/23, vedenie: Zuzana Špitálová}\\
	{\small \texttt{xlovasl@stuba.sk}}\\
	}

\date{\small 6. november 2022} % upravte



\begin{document}

\maketitle

\begin{abstract}
V dnešnej dobe je pre mládež populárnou témou hranie hier. Z tohoto dôvodu vznikol smer výskumu, ktorý sa zameriava na využitie hier pri výučbe a učení v školách, pričom sa kladie dôraz na efektivitu výučby a učenia. V mojom článku sa chcem venovať niekoľko programom určených práve pre školy a študentov na zefektívnenie učenia, pričom mojim hlavným cieľom je porovnať konkrétne hry a ich vplyv na efektivitu učenia a výučby programovania. Výstupom bude prehľadový článok porovnávajúci konktrétne hry, cez vybrané aspekty. Medzi vybrané hry, ktoré som sa rozhodol analyzovať, patrí Clara's World\cite{ClarasWorld}, jednoducho vyzerajúca hra ktorá má hráčov oboznámiť so základnou syntaxou programovania, znázorňovaného pomocou Clary - herného modelu lienky, ktorý plní zadané príkazy pohybovaním sa po hernej ploche. ClaraQuest\cite{ClaraQuest} je modernejšia verzia, ktorá podporuje VR platformu, pričom bola vyvinutá v rámci výskumu na FIIT STU. Ďaľšou skúmanou hrou je Cubely\cite{Cubely}, ktorá sa taktiež zameriava na podporu výučby objektovo orientovaného programovania. A poslednou skúmanou hrou je Minecraft: Education\cite{MinecraftEd}, ktorá vznikla z populárnej hry Minecraft. Podporované hry využívajú aj aspekty virtuálnej reality, čím sa stávajú lepšie porovnatelnými medzi sebou. V závere článku zhodnotím výhody a nevýhody jednotlivých hier a sformulujem špecifické odporúčania pre zefektívnenie učenia a výučby. 

\end{abstract}




\section{Analýza oblasti}
Už pár rokov je, hlavne v Amerike a mnoho progresívnych krajinách Európy, v oblasti vzdelávania populárna téma hier vo výučbe. Samotné využitie rôznych hier a herných konceptov sa už v niekoľko školách aktívne využíva, no iné školy preferujú klasické vyučovanie. Hlavným zámerom týchto hier je zjednodušiť študentom pohľad na učivo ktoré sa učia, a tým aj jeho lepšie zapamätanie a naučenie sa. Moderné variácie týchto hier sú využívané pomocou technológie virtuálnej reality. Virtuálna realita funguje pomocou špeciálnych okuliarov, ktoré premietajú hru pre používateľa, ako keby sa on sám v nej nachádzal. Zakomponovanie virtuálnej reality do gamifikácie v školách otvára širšie možnosti pre výučbu a taktiež aj vyzerá pre študentov oveľa atraktívnejšie, a tým podporuje chuť sa učiť a chápať učivu. Na Slovensku je gamifikácia v školách aj napriek množstvo projektom a pokusom o jej zviditeľnenie pomerne zanedbaná, a jej koncepty využíva len malé množstvo škôl.

\section{Existujúce projekty}
V tejto sekcii si spomenieme jednotlivé projekty, ktoré neskôr budú v podsekciách predstavené a budeme sa im venovať do hĺbky. Prvou hrou je projekt Cubely \cite{Cubely}, ktorá je celá navrhnutá vo virtuálnej realite. Využíva koncept block-based programovania a učí používateľa základné koncepty objektovo orientovaného programovania. Ďalšou je hra Clara's World, desktopová hra ktorá je zrealizovaná v grafickom prostredí s vizualizáciou kódu, a sprevádza používateľa cez jazyk Java. Ku hre Clara's World bola vytvorená aj nadstavba na virtuálnu realitu, a to ClaraQuest, hra ktorá má ten istý v koncept, no jej ovládanie a prostredie je realizované vo virtuálnej realite. Poslednou spomenutou hrou je Minecraft: Education, upravená variácia populárnej hry Minecraft, ktorá slúži na vyučovanie mnohých, či už prírodovedných predmetov ako biológia alebo chémia, no aj humanitných predmetov ako dejepis či výučba jazykov. Hra je postavená na princípe výučby predmetov v atraktívnom prostredí hry Minecraft, ktorá má nespočetne veľa mladých fanúšikov.

\subsection{Cubely}
 Cubely je hra vo virtuálnej realite, ktorá vytvára jednoduché a zaujímavo vyzerajúce prostredie pre používateľa. Je to forma programovacieho jazyka, kde používateľ tvorí jeho vlastné riešenie na poskytnuté problémy pomocou vizuálnych interpretácií kódu v kockách, alebo „blockov“. Cubely využíva koncept block-based programovania - to znamená, že užívateľ namiesto písania kódu skladá dokopy „blocky“, ktoré obsahujú príkazy a tým tvorí svoj vlastný kód zjednodušeným spôsobom. Cieľom každého levelu v hre je vytvoriť taký kód, ktorý korektne dovedie cvičnú postavičku na určené miesto. Na tvorbu hry bolo využité vývojárske prostredie Unity a programovací jazyk CSharp. Cvičenia, ktoré Cubely poskytuje na rozvíjanie programátorských schopností sú veľmi podobné zadaniam z Code.org. Celý svet Cubely je založený na téme Minecraft, aby spríjemnila a uľahčila študentom na školách jej používanie tým, že sa ocitnú v prostredí populárnej hry. Cubely bolo vytvorené Martinom Hoangom, študentom FIIT, a hra bola veľkou častou jeho bakalárskej práce.

\subsection{Clara's World}

\subsection{ClaraQuest}

\subsection{Minecraft: Education}
Minecraft: Education je variant populárneho Minecraft-u, hry ktorá je známa medzi mnohými. Je to verzia hry ktorá sa využíva primárne na výučbu množstva predmetov na základnej škole. Pôvodne bola navrhnutá na výučbu programovania, no po veľkom úspechu sa vývojári rozhodli pridať herné leveli, ktoré ukazujú študentom učivo prostredníctvom sveta Minecraftu. Týmto spôsobom sa dokážu študenti učiť na predmety  zatiaľ čo objavujú svet hry. Hra svojimi levelmi rozvíja tvorivosť, schopnosť riešenia problémov, kritické myslenie, či komunikáciu v tíme. Príklady využitia sú napríklad modul Zostavovač kódu, v ktorom študenti programujú postavičku „Agenta“, ktorý vykonáva správanie, ktoré mu študent zadá inštrukciami cez jednoduché programátorské prostredie. Ďalším príkladom je bonusový obsah, ktorý tvorcovia pridali v roku 2018, ktorý do hry pridáva rôzne prvky na vyučovanie chémie: materiály, nástroje a rôzne látky, s ktorými môže študent experimentovať, kombinovať elementy, vytvárať látky z iných látok a ďalšie. Týmto spôsobom sa študent rýchlo a efektívne dostane do tématiky chémie. Hra bola vytvorená v súkromnom vývojárskom prostredí a využíva Javu. Minecraft: Education bol vytvorený spoločnostou Mojang Studios, vývojárskou firmou založenou na tvorbe hier. 

\section{Porovnanie projektov}

\section{Vyhodnotenie}

\section{Záver}
V súčasnosti je článok rozpracovaný. V ďalších častiach sa budem venovať samotnému porovnaniu jednotlivých hier a vyhodnoteniu, ktorá je kedy vhodná na použitie vzhľadom k téme učenia a používateľom hier. Chcem poďakovať Ing. Lukášovi Grafovi za konzultácie k danej téme.


% týmto sa generuje zoznam literatúry z obsahu súboru literatura.bib podľa toho, na čo sa v článku odkazujete
\bibliography{sample}
\bibliographystyle{plain} % prípadne alpha, abbrv alebo hociktorý iný
\end{document}
